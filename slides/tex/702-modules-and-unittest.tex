\documentclass[aspectratio=169]{beamer}
\usetheme{metropolis}

% Packages
\usepackage{hyperref}
\usepackage{amsmath, amssymb, bm, graphicx}
\usepackage{xcolor}
\hypersetup{hidelinks}
\DeclareMathOperator*{\argmin}{arg\,min}
\DeclareMathOperator*{\argmax}{arg\,max}

% Define blood red color and redefine emph
\definecolor{bloodred}{HTML}{CC0000}
\renewcommand{\emph}[1]{\textcolor{red}{#1}}

% Document Metadata
\title{Modules and unittest}
\author{Joachim Vandekerckhove}
\date{Winter 2025}

\begin{document}

\maketitle


\begin{frame}[fragile]{Basic Test Structure in Python}
    \begin{itemize}
        \item Tests are organized in classes inheriting from \texttt{unittest.TestCase}
        \item Each test method starts with \texttt{test\_}
        \item Sometimes it is helpful to add a \texttt{setUp()} method
    \end{itemize}
    \begin{verbatim}
class TestCalculator(unittest.TestCase):
    def setUp(self):
        self.calc = Calculator()

    def test_add(self):
        self.assertEqual(self.calc.add(2, 3), 5)
        self.assertEqual(self.calc.add(-1, 1), 0)
    \end{verbatim}
\end{frame}

\begin{frame}{Assert Methods}
    Common assertions in unittest:
    \begin{itemize}
        \item \texttt{assertEqual(a, b)} - verify a equals b
        \item \texttt{assertAlmostEqual(a, b)} - for floating point
        \item \texttt{assertRaises(Error)} - verify exceptions
        \item \texttt{assertTrue(x)} - verify x is True
        \item \texttt{assertFalse(x)} - verify x is False
    \end{itemize}
\end{frame}

\begin{frame}[fragile]{Testing Edge Cases}
    Example: Testing division by zero
    \begin{verbatim}
    def test_divide_by_zero(self):
        with self.assertRaises(ValueError):
            self.calc.divide(5, 0)
    \end{verbatim}
    \begin{itemize}
        \item Test both normal and edge cases
        \item Test invalid inputs
        \item Test boundary conditions
    \end{itemize}
\end{frame}

\begin{frame}[fragile]{Testing Complex Behavior}
    Example: Testing operation history
    \begin{verbatim}
    def test_history(self):
        self.calc.add(2, 3)
        self.calc.multiply(4, 5)
        history = self.calc.get_history()
        self.assertEqual(len(history), 2)
    \end{verbatim}
    \begin{itemize}
        \item Test state changes
        \item Test sequences of operations
        \item Verify side effects
    \end{itemize}
\end{frame}

\begin{frame}{Best Practices}
    \begin{itemize}
        \item Write tests before or while writing code (TDD)
        \item One test method per feature/scenario
        \item Clear, descriptive test names
        \item Independent tests (no dependencies between tests)
        \item Use \texttt{setUp()} for common initialization
        \item Keep tests simple and focused
    \end{itemize}
\end{frame}

\begin{frame}[fragile]{Running Tests}
    \begin{itemize}
        \item Run all tests in the project:
        \begin{verbatim}
    python -m unittest discover -s tests
        \end{verbatim}
        \item Run a specific test file:
        \begin{verbatim}
    python -m unittest tests.test_operations
        \end{verbatim}
        \item Run a specific test case:
        \begin{verbatim}
    python -m unittest tests.test_operations.TestClass.test_method
        \end{verbatim}
    \end{itemize}
\end{frame}

\begin{frame}{Project Structure for Tests}
    Key requirements:
    \begin{itemize}
        \item Tests should be in a \texttt{tests} directory
        \item Include empty \texttt{\_\_init\_\_.py} in \texttt{tests} directory
        \item Test files should follow naming pattern \texttt{test\_*.py}
        \item Run tests from project root directory
    \end{itemize}
\end{frame}

\maketitle

\end{document}

