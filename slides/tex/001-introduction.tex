\documentclass[aspectratio=169]{beamer}
\usetheme{metropolis}

% Packages
\usepackage{amsmath,amssymb}
\usepackage{hyperref}
\hypersetup{hidelinks}

% Document Metadata
\title{Computational Lab Skills for Cognitive Science}
\author{Joachim Vandekerckhove}
\date{Winter 2025}

\begin{document}

\maketitle
\begin{frame}{Course Description}
This course provides an in-depth introduction to skills needed to conduct projects of a computational nature, with a focus on research in cognitive sciences.\pause

\textbf{Topics Covered}:
\begin{itemize}[<+->]
    \item Computational reproducibility
    \item Version control
    \item Writing high-quality, maintainable code
    \item Numerical simulations
    \item Model fitting
    \item Numerical methods for integration and optimization
\end{itemize}
\end{frame}

\begin{frame}{Tools}
\textbf{Tools and Environment}:
\begin{itemize}[<+->]
    \item Docker Desktop
    \item Ubuntu Linux environment
    \item Python programming language
\end{itemize}
\end{frame}

\begin{frame}{Prerequisites}
\begin{itemize}[<+->]
    \item (PSYC 14M or PSYC 114M or COGS 14P or ICS 31)
    \item AND (PSYC 10C or STAT 7 or STAT 110)
\end{itemize}
\end{frame}

\begin{frame}{Flipped Lectures}
\begin{itemize}[<+->]
\item[]
\textbf{Recorded Lectures}:
\begin{itemize}
    \item Published Monday afternoons
    \item Review before class time
\end{itemize}
\item[]
\textbf{In-Person Meetings}:
\begin{itemize}
    \item Wednesdays, 4:00 PM--4:50 PM
\end{itemize}
\item[]
\textbf{Virtual Meetings} (Optional):
\begin{itemize}
    \item Fridays, 4:00 PM--4:50 PM
\end{itemize}
\end{itemize}
\end{frame}

\begin{frame}{Assessment}
\textbf{Individual Assignments}:
\begin{itemize}[<+->]
    \item Done individually
    \item Graded based on meeting all requirements by the deadline
    \item Grade of `A` requires all assignments on time and functional
\end{itemize}
\end{frame}

\begin{frame}{Assessment}
\textbf{Group Assignments}:
\begin{itemize}[<+->]
    \item Work in groups of 3--5
    \item Notify the instructor of group members by the end of January
    \item Grading primarily based on group product quality
    \item Individual contributions evaluated via GitHub commits if needed
\end{itemize}
\end{frame}

\begin{frame}{Assessment}
\textbf{Submission Policy}:
\begin{itemize}[<+->]
    \item No late submissions (assignments due at noon on the due date)
    \item GitHub snapshot taken at the deadline
    \item Recommendation: Set an internal deadline a day earlier than the due date
\end{itemize}
\end{frame}

\begin{frame}{Week 2: Computational Reproducibility}
Learn how to create reproducible computational environments using containers.

Making sure your computations can be reproduced is a foundational skill that---among other things---ensures that analyses can be replicated reliably.
\begin{itemize}[<+->]
    \item Assessment: Create a container and run Python in it
\end{itemize}
\end{frame}

\begin{frame}{Week 3: Version Control}
This week covers version control basics using GitHub, focusing on building a workflow that includes smooth and secure interaction with GitHub from a virtual environment.
\begin{itemize}[<+->]
    \item Assessment: Pull/push to GitHub from a container
\end{itemize}
\end{frame}

\begin{frame}{Week 4: Object-Oriented Programming}
Explore how to design modular and reusable code using object-oriented programming. Focus on creating objects tailored to specific tasks.
\begin{itemize}[<+->]
    \item Assessment: Write an object according to specifications
\end{itemize}
\end{frame}

\begin{frame}{Week 5: No Class}
A lighter week you can use for catching up and reviewing material, or focus on your midterms in other classes.  Extra time for last week's assignment.
\end{frame}

\begin{frame}{Week 6: Test-Driven Development}
Learn to write robust, maintainable code by using test-driven development. Write tests first to guide your coding process. Test-driven development as a way of life.
\begin{itemize}[<+->]
    \item Assessment: Develop specifications for a project
\end{itemize}
\end{frame}

\begin{frame}{Week 7: Optimization}
Understand and apply optimization techniques to solve computational and statistical problems.
\begin{itemize}[<+->]
    \item Assessment: TBD
\end{itemize}
\end{frame}

\begin{frame}{Week 8: Code Smells and Refactoring}
Learn to identify ``code smells'' that indicate poor design and how to refactor for cleaner, more maintainable code.
\begin{itemize}[<+->]
    \item Assessment: Identify code smells and refactor as needed
\end{itemize}
\end{frame}

\begin{frame}{Week 9: Simulation}
Design and conduct numerical experiments using simulation techniques. Gain insights into the behavior of computational models.
\begin{itemize}[<+->]
    \item Assessment: Design and conduct a numerical experiment
\end{itemize}
\end{frame}

\begin{frame}{Week 10: Integration \& MCMC}
Explore numerical integration and Markov Chain Monte Carlo methods for solving complex problems in cognitive science.
\begin{itemize}[<+->]
    \item Assessment: TBD
\end{itemize}
\end{frame}

\begin{frame}{Academic Dishonesty}
There is no tolerance for academic dishonesty or fraud. Any form of fraud designed to circumvent course policies will result in a failing grade. The professor makes no judgment calls regarding academic dishonesty. Any academic dishonesty, no matter how small, will be escalated to academic authorities.
\end{frame}

\begin{frame}{Resources}
\textbf{Disability Services}: {https://dsc.uci.edu/}{}

\textbf{Academic Dishonesty}: {https://aisc.uci.edu/students/academic-integrity/index.php}{}

\textbf{Copyright Policy}: {http://copyright.universityofcalifornia.edu/use/teaching.html}{}
\end{frame}

\begin{frame}{Key Philosophy: Figure It Out}
\begin{overlayarea}{1\textwidth}{.5\textheight}
If you obtain a single skill in this class, I hope it is \textbf{Figuring It Out.}\\[3ex]
\only<2>{
    \begin{quote}
    The new education must teach the individual how to classify and reclassify information, how to evaluate its veracity, how to change categories when necessary, how to move from the concrete to the abstract and back, how to look at problems from a new direction—how to teach himself. Tomorrow’s illiterate will not be the man who can’t read; he will be the man who has not learned how to learn. (-- Herbert Gerjuoy)
    \end{quote}
}%
\only<3->{
\begin{itemize}
\item{The first few weeks will focus on setting up safe sandboxes in which you can screw up your code or even your operating system.}
\only<4->{\item Once that is set up, you should get into a habit of Googling for solutions or asking an AI and experimenting and trying things out to see if they work.}
\only<5->{\item We will talk a lot about testing your own code and software.}
\only<6->{\item Finding partial solutions from unverified sources, implementing them, and then conducting rigorous tests is a highly generalizable coding paradigm.}
\end{itemize}
}
\end{overlayarea}
\end{frame}

\begin{frame}{AI Policy}
\begin{itemize}[<+->]
    \item[]
    \textbf{General AI Policy:}
\begin{itemize}[<+->]
    \item Use of tools like Copilot, ChatGPT, and generative AI is permitted.
    \item AI is a productivity enhancer but not a substitute for knowing what you are doing.
\end{itemize}
\item[]\textbf{Rules}:
\begin{enumerate}[<+->]
    \item Students are individually responsible for all submissions.
    \item Acknowledge use of reference works, websites, and AI tools in comments.
\end{enumerate}
\end{itemize}
\end{frame}

\maketitle

\end{document}
