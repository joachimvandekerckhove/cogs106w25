\documentclass[aspectratio=169]{beamer}
\usepackage{pgf}
\usepackage{tikz}
\usetheme{metropolis}

% Packages
\usepackage{hyperref}
\usepackage{amsmath, amssymb, bm, graphicx}
\usepackage{xcolor}
\usepackage{listings}
\hypersetup{hidelinks}
 
% Configure listings for Python code
\lstset{
    language=Python,
    basicstyle=\ttfamily\small,
    keywordstyle=\color{blue},
    stringstyle=\color{red},
    showstringspaces=false,
    frame=tb,
    rulecolor=\color{palegreen},
    backgroundcolor=\color{paleyellow},
    xleftmargin=2.5mm,
    xrightmargin=2.5mm,
    framexleftmargin=2.5mm,    % internal left padding
    framexrightmargin=2.5mm,   % internal right padding
    framesep=2.5mm,            % vertical padding (top and bottom)
    numbers=none,
    breaklines=true,
    breakatwhitespace=true,
    escapeinside={(*@}{@*)},
    basicstyle=\ttfamily\scriptsize
}

\definecolor{paleyellow}{rgb}{0.98,0.98,0.9}
\definecolor{brickred}{rgb}{0.8,0.2,0.2}
\definecolor{palegreen}{rgb}{0.9,0.98,0.9}
\renewcommand{\emph}[1]{\textcolor{red}{#1}}

% Document Metadata
\title{Error handling}
\author{Joachim Vandekerckhove}
\date{Winter 2025}

\begin{document}

\frame{\titlepage}

% Slide: Introduction
\begin{frame}{Introduction to Error Handling}
    \textbf{Why is error handling important?}
    \begin{itemize}[<+->]
        \item Errors are inevitable in programming
        \item Proper handling improves software robustness and usability
        \item Helps with debugging and logging
        \item Good error handling prevents crashes
        \item Most importantly, it prevents unexpected behavior
    \end{itemize}
\end{frame}

% Slide: Overview of Error Types
\begin{frame}{Categories of Errors in Python}
    \begin{itemize}
        \item<1-> \textbf{Syntax Errors}
        \begin{itemize}[<2->]
            \item Caught before execution
            \item Invalid Python syntax
        \end{itemize}
        \item<3-> \textbf{Runtime Errors (Exceptions)}
        \begin{itemize}[<4->]
            \item Occur during program execution
            \item Many types: ZeroDivisionError, TypeError, IndexError, etc.
            \item Can be caught and handled
        \end{itemize}
        \item<5-> \textbf{Logical Errors}
        \begin{itemize}[<6->]
            \item Program runs but produces wrong results
            \item Hardest to detect -- no error messages
            \item Require careful testing and debugging
        \end{itemize}
    \end{itemize}
\end{frame}

% Slide: Syntax Errors
\begin{frame}[fragile]{Syntax Errors}
    \textbf{Example of a Syntax Error:}
    \pause
    \begin{lstlisting}
>>> print "Hello"
  File "<stdin>", line 1
    print "Hello"
    ^^^^^^^^^^^^^
SyntaxError: Missing parentheses in call to 'print'. Did you mean print(...)?
    \end{lstlisting}
\end{frame}

% Slide: Logical Errors
\begin{frame}[fragile]{Logical Errors}
    \textbf{Example of a Logical Error:}
    \pause
    \begin{lstlisting}
>>> # Task: Average the first five numbers
>>> total = 0
>>> for i in range(5):
...     total += i
>>> print(f"Average: {total/5}")
Average: 2.0
>>> # Bug: off by one
    \end{lstlisting}
\end{frame}

% Slide: Runtime Error - ZeroDivisionError
\begin{frame}[fragile]{Runtime Errors}
    \texttt{ZeroDivisionError}
    \begin{itemize}[<+->]
        \item Occurs when dividing by zero
        \item Common in mathematical calculations
    \end{itemize}
    \pause
    \begin{lstlisting}
>>> 10 / 0
Traceback (most recent call last):
  File "<stdin>", line 1, in <module>
ZeroDivisionError: division by zero
    \end{lstlisting}
\end{frame}

% Slide: Runtime Error - IndexError
\begin{frame}[fragile]{Runtime Errors}
    \texttt{IndexError}
    \begin{itemize}[<+->]
        \item Occurs when accessing a non-existent index
        \item Common when working with sequences (lists, tuples, etc.)
    \end{itemize}
    \pause
    \begin{lstlisting}
>>> nums = [1, 2, 3]
>>> nums[5]
Traceback (most recent call last):
  File "<stdin>", line 1, in <module>
IndexError: list index out of range
    \end{lstlisting}
\end{frame}

% Slide: Runtime Error - TypeError
\begin{frame}[fragile]{Runtime Errors}
    \texttt{TypeError}
    \begin{itemize}[<+->]
        \item Occurs when an operation is performed on incompatible types
        \item Common when mixing different data types
    \end{itemize}
    \pause
    \begin{lstlisting}
>>> "hello" + 42
Traceback (most recent call last):
  File "<stdin>", line 1, in <module>
TypeError: can only concatenate str (not "int") to str

>>> len(42)
Traceback (most recent call last):
  File "<stdin>", line 1, in <module>
TypeError: object of type 'int' has no len()
    \end{lstlisting}
\end{frame}


% Slide: Common Python Exceptions Table
\begin{frame}{Common Python Exceptions}
    \begin{columns}
        \begin{column}{0.48\textwidth}
            \begin{table}
            \small
            \begin{tabular}{@{}ll@{}}
                \textbf{Exception} & \textbf{Description} \\ \hline
                SyntaxError & Invalid syntax \\
                TypeError & Wrong type \\
                ValueError & Invalid value \\
                IndexError & Bad sequence index \\
                KeyError & Key not found \\
                NameError & Name not found \\
                AttributeError & Missing attribute \\
                ImportError & Import failed \\ \hline
            \end{tabular}
            \end{table}
        \end{column}
        \begin{column}{0.48\textwidth}
            \begin{table}
            \small
            \begin{tabular}{@{}ll@{}}
                \textbf{Exception} & \textbf{Description} \\ \hline
                FileNotFoundError & File not found \\
                IOError & I/O operation failed \\
                ZeroDivisionError & Division by zero \\
                AssertionError & Assert failed \\
                RuntimeError & Generic error \\
                OverflowError & Value too large \\
                MemoryError & Out of memory \\
                OSError & System error \\ \hline
            \end{tabular}
            \end{table}
        \end{column}
    \end{columns}
\end{frame}


% Detailed Exception Slides - Input/Output
\begin{frame}{I/O Related Exceptions}
    \onslide<1->{\textbf{FileNotFoundError}}
    \begin{itemize}[<2->]
        \item File path doesn't exist
        \item Common in: file operations, config loading, data processing
        \item Example: \texttt{open('missing.txt', 'r')}
    \end{itemize}

    \onslide<5->{\textbf{IOError}}
    \begin{itemize}[<6->]
        \item General input/output failures
        \item Disk full, pipe broken, network timeout
        \item Example: writing to a closed file
    \end{itemize}

\end{frame}

% Data Structure Exceptions
\begin{frame}{Data Structure Exceptions}
    \onslide<1->{\textbf{IndexError}}
    \begin{itemize}[<2->]
        \item Accessing sequence beyond bounds
        \item Example: \texttt{alphabet[27]} on a list of size 26
    \end{itemize}

    \onslide<4->{\textbf{KeyError}}
    \begin{itemize}[<5->]
        \item Missing dictionary key
        \item Example: \texttt{dict['missing\_key']}
    \end{itemize}

    \onslide<7->{\textbf{AttributeError}}
    \begin{itemize}[<8->]
        \item Accessing non-existent object attribute
        \item Example: \texttt{sdt.flaseAlarms()}
    \end{itemize}
\end{frame}

% Type and Value Exceptions
\begin{frame}{Type and Value Exceptions}
    \onslide<1->{\texttt{TypeError}}
    \begin{itemize}[<2->]
        \item Operation on incompatible types
        \item Example: \texttt{"U" + 2}
    \end{itemize}

    \onslide<4->{\texttt{ValueError}}
    \begin{itemize}[<5->]
        \item Right type but invalid value
        \item Common in: parsing, conversion operations
        \item Example: \texttt{int("not a number")}
    \end{itemize}

    \onslide<8->{\texttt{NameError}}
    \begin{itemize}[<9->]
        \item Using undefined variable
        \item Example: typos, scope issues, using variable before assignment
    \end{itemize}
\end{frame}

% Resource and System Exceptions
\begin{frame}{Resource and System Exceptions}
    \onslide<1->{\texttt{MemoryError}}
    \begin{itemize}[<2->]
        \item Out of memory
        \item Example: creating huge lists/arrays, infinite recursion
    \end{itemize}

    \onslide<4->{\texttt{OverflowError}}
    \begin{itemize}[<5->]
        \item Arithmetic operation too large
        \item Example: very large exponential operations
    \end{itemize}

    \onslide<7->{\texttt{ImportError}}
    \begin{itemize}[<8->]
        \item Module import fails
        \item Example: importing non-installed package
    \end{itemize}
\end{frame}

% Special Cases
\begin{frame}{Special Case Exceptions}
    \onslide<1->{\texttt{ZeroDivisionError}}
    \begin{itemize}[<2->]
        \item Division or modulo by zero
        \item Example: \texttt{x / 0} or \texttt{y \% 0}
    \end{itemize}

    \onslide<4->{\texttt{AssertionError}}
    \begin{itemize}[<5->]
        \item Assert statement fails
    \end{itemize}

    \onslide<6->{\texttt{RuntimeError}}
    \begin{itemize}[<7->]
        \item Base class for many custom exceptions
        \item When no other error type will do
    \end{itemize}
\end{frame}

% Slide: Try-Except Blocks
\begin{frame}[fragile]{Handling Exceptions: Try-Except Blocks}
    \textbf{Basic Syntax}
    \pause
    \begin{lstlisting}
try:
    x = 1 / 0
except ZeroDivisionError:
    print("Cannot divide by zero!")
    \end{lstlisting}
    
    \pause
    \textbf{Why is this important?}
    \begin{itemize}[<+->]
        \item Prevents crashes and allows graceful failure handling
        \item Allows debugging by catching and logging errors
    \end{itemize}
    \pause
    \emph{Do not use try-except to continue execution unless you know exactly what is going on!}
\end{frame}

% Slide: Else and Finally
\begin{frame}[fragile]{Else and Finally Blocks}
    \textbf{Using Else and Finally}
    \begin{itemize}[<+->]
        \item \texttt{else}: Runs only if no exception occurs
        \item \texttt{finally}: Runs regardless of whether an exception occurred or not
    \end{itemize}

    \pause
    \textbf{Example}
    \pause
    \begin{lstlisting}
try:
    num = int(input("Enter a number: "))
except ValueError:
    print("Invalid input!")
else:
    print("You entered:", num)
    \end{lstlisting}

    \pause
    (What else can age not be?)
\end{frame}

% Avoid defensive programming
\begin{frame}[fragile]{Avoid defensive programming!}
    \begin{itemize}[<+->]
        \item A practice you should avoid is to use \texttt{else:} to 
        continue execution notwithstanding the fact that an 
        exception occurred
        
        \item If the user supplied invalid input,
        you should error and tell them what they did wrong
    \end{itemize}

    \onslide<3->{\emph{Avoid defensive programming!}}
\end{frame}

% Catch multiple possible exceptions
\begin{frame}[fragile]{Catch multiple possible exceptions}
    \begin{lstlisting}
try:
    num = int(input("Enter a number: "))
    if num < 0:
        raise ValueError("Number cannot be negative")
    if num > 100:
        raise ValueError("Number must be 100 or less")
except ValueError as e:
    print(f"Invalid number: {e}")
    raise
except TypeError as e:
    print(f"Wrong type of input: {e}")
    raise
except Exception as e:  # Catch any unexpected errors
    print(f"Unexpected error: {e}")
    raise

print("You entered:", num)
return num  # Actually use the validated input
    \end{lstlisting}
\end{frame}

% Slide: Raising Exceptions
\begin{frame}[fragile]{Raising Exceptions}
    \textbf{Why Raise Exceptions?}
    \begin{itemize}
        \item To signal an error when a function receives invalid input.
        \item Makes debugging easier by identifying issues early.
    \end{itemize}

    \textbf{Example}
    \begin{lstlisting}
def check_age(age):
    if age < 0:
        raise ValueError("Age cannot be negative!")
    \end{lstlisting}
\end{frame}


\begin{frame}[fragile]{Using Appropriate Exceptions: Examples}
    \textbf{Example: Input Validation}
    
    Input validation is convenient in separate methods.
    \pause
    \begin{lstlisting}
def process_age(age_str):
    try:
        age = int(age_str)  # ValueError if not a number
        if age < 0:
            raise ValueError("Age cannot be negative")
        if age > 150:
            raise ValueError("Age seems unrealistic")
        return age # Everything went well
    except ValueError as e:
        print(f"Invalid age: {e}")
        raise
    except Exception as e:  # Catch any unexpected errors
        print(f"Unexpected error processing age: {e}")
        raise
    \end{lstlisting}
\end{frame}

\begin{frame}[fragile]{Using Appropriate Exceptions: More Examples}
    \textbf{Example: File Operations}
    \pause
    \begin{lstlisting}
def read_config(filename):
    try:
        with open(filename) as f:  # FileNotFoundError
            data = json.loads(f.read())  # JSONDecodeError
            return data['settings']  # KeyError
    except FileNotFoundError:
        print("Config file missing")
        raise  # Re-raise the current exception
    except json.JSONDecodeError:
        print("Invalid JSON format")
        raise
    except KeyError:
        print("Missing 'settings' in config")
        raise
    except Exception as e:
        print(f"Unexpected error: {e}")
        raise
    \end{lstlisting}
\end{frame}

\begin{frame}{When to Use Custom Exceptions}
    \textbf{Create Custom Exceptions When:}
    \begin{itemize}
        \item Built-in exceptions don't clearly convey the error
        \item You need domain-specific error handling
        \item You want to group related errors
        \item You need to add custom error attributes
    \end{itemize}

    \textbf{Examples of Good Custom Exceptions:}
    \begin{itemize}
        \item \texttt{DatabaseConnectionError}
        \item \texttt{InvalidConfigurationError}
        \item \texttt{UserAuthenticationError}
        \item \texttt{APIRateLimitExceeded}
    \end{itemize}
\end{frame}

\begin{frame}[fragile]{Creating and Using Custom Exceptions}
    \textbf{Example: Custom Exception Class}
    \pause
    \begin{lstlisting}
class ConfigError(Exception):
    def __init__(self, message, missing_keys=None):
        self.missing_keys = missing_keys or []
        super().__init__(message)

def validate_config(config):
    required = ['api_key', 'host', 'port']
    missing = [k for k in required if k not in config]
    if missing:
        raise ConfigError("Missing required keys", missing)
    \end{lstlisting}
\end{frame}


% Slide: Logging Errors
\begin{frame}[fragile]{Logging Errors Instead of Printing}
    \textbf{Why Use Logging?}
    \begin{itemize}[<+->]
        \item Avoids cluttering the console.
        \item Keeps a permanent record of errors.
        \item Helps with debugging and monitoring in production.
    \end{itemize}

    \pause
    \textbf{Example}
    \pause
    \begin{lstlisting}
import logging
logging.basicConfig(filename="errors.log", level=logging.ERROR)
try:
    1 / 0
except ZeroDivisionError as e:
    logging.error(f"Error: {e}")
    \end{lstlisting}
\end{frame}

% Slide: Testing Exception Handling
\begin{frame}[fragile]{Testing Exception Handling}
    \textbf{Using unittest}
    \pause
    \begin{lstlisting}
import unittest

def divide(a, b):
    if b == 0:
        raise ZeroDivisionError("Cannot divide by zero!")
    return a / b

class TestDivide(unittest.TestCase):
    def test_zero_division(self):
        with self.assertRaises(ZeroDivisionError):
            divide(1, 0)
    \end{lstlisting}
\end{frame}

% Slide: Best Practices
\begin{frame}{Best Practices for Error Handling}
    \textbf{Dos and Don'ts}
    \begin{itemize}[<+->]
        \item \textbf{Catch specific exceptions} instead of using a broad \texttt{Exception}
        \item \textbf{Use logging} instead of \texttt{print()} to track errors
        \item \textbf{Use \texttt{finally}} to clean up resources like file handles or database connections
        \item \textbf{Don't suppress exceptions} without handling them properly
        \item \textbf{Use custom exceptions} for better error categorization
        \item \textbf{Write tests} to verify that exception handling works correctly
    \end{itemize}
\end{frame}

\maketitle

\end{document}
